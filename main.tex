% Exemplo de relatório técnico do IC
% Criado por P.J.de Rezende antes do Alvorecer da História.
% Modificado em 97-06-15 e 01-02-26 por J.Stolfi.
% Last edited on 2003-06-07 21:12:18 by stolfi
% modificado em 1o. de outubro de 2008
% modificado em 2012-09-25 para ajustar o pacote UTF8. Contribuicao de
%   Rogerio Cardoso

\documentclass[11pt,twoside]{article}
\usepackage{techrep-PFG-ic}

%%% SE USAR INGLÊS, TROQUE AS ATIVAÇÕES DOS DOIS COMANDOS A SEGUIR:
\usepackage[brazil]{babel}
%% \usepackage[english]{babel}

%%% SE USAR CODIFICAÇÃO LATIN1, TROQUE AS ATIVAÇÕES DOS DOIS COMANDOS A
%%% SEGUIR:
%% \usepackage[latin1]{inputenc}
\usepackage[utf8]{inputenc}

\begin{document}

%%% PÁGINA DE CAPA %%%%%%%%%%%%%%%%%%%%%%%%%%%%%%%%%%%%%%%%%%%%%%%
% 
% Número do relatório
\TRNumber{01}

% DATA DE PUBLICAÇÃO (PARA A CAPA)
%
\TRYear{16}  % Dois dígitos apenas
\TRMonth{12} % Numérico, 01-12

% LISTA DE AUTORES PARA CAPA (sem afiliações).
\TRAuthor{A. Goodbeer \and Ch. Opps \and E. S. Puma}

% TÍTULO PARA A CAPA (use \\ para forçar quebras de linha).
\TRTitle{Comparação\\de cervejas brasileiras}

\TRMakeCover

%%%%%%%%%%%%%%%%%%%%%%%%%%%%%%%%%%%%%%%%%%%%%%%%%%%%%%%%%%%%%%%%%%%%%%
% O que segue é apenas uma sugestão - sinta-se à vontade para
% usar seu formato predileto, desde que as margens tenham pelo
% menos 25mm nos quatro lados, e o tamanho do fonte seja pelo menos
% 11pt. Certifique-se também de que o título e lista de autores
% estão reproduzidos na íntegra na página 1, a primeira depois da
% página de capa.
%%%%%%%%%%%%%%%%%%%%%%%%%%%%%%%%%%%%%%%%%%%%%%%%%%%%%%%%%%%%%%%%%%%%%%

%%%%%%%%%%%%%%%%%%%%%%%%%%%%%%%%%%%%%%%%%%%%%%%%%%%%%%%%%%%%%%%%%%%%%%
% Nomes de autores ABREVIADOS e titulo ABREVIADO,
% para cabeçalhos em cada página.
%
\markboth{Goodbeer, Opps e Puma}{Cervejas Brasileiras}
\pagestyle{myheadings}

%%%%%%%%%%%%%%%%%%%%%%%%%%%%%%%%%%%%%%%%%%%%%%%%%%%%%%%%%%%%%%%%%%%%%%
% TÍTULO e NOMES DOS AUTORES, completos, para a página 1.
% Use "\\" para quebrar linhas, "\and" para separar autores.
%
\title{Comparação de Cervejas Brasileiras}

\author{Arthur Goodbeer\thanks{Beernautics, Inc.} \and
Charles Opps\thanks{Institute for Advanced Salaries, 
St. Anford University, 95307 Las Cervezas, CA} \and
Eduardo dos Santos Puma\thanks{Instituto  de Computação, Universidade
Estadual  de Campinas, 13081-970  Campinas,  SP.  Pesquisa desenvolvida com
suporte financeiro parcial do CNPq, processo 123456/90-x}}

\date{}

\maketitle

%%%%%%%%%%%%%%%%%%%%%%%%%%%%%%%%%%%%%%%%%%%%%%%%%%%%%%%%%%%%%%%%%%%%%%

\begin{abstract} 
  Este trabalho é um relatório parcial de um projeto temático 
  plurianual, visando o estudo comparativo de diversas espécies de
  cervejas nativas do sub-continente brasileiro.  A realização deste
  trabalho contou com o suporte financeiro do CNPq e FAPESP, e foi
  imensamente facilitada pela infraestrutura de pesquisa cervisíaca
  instalada no Campus da UNICAMP.

  Com base nessas pesquisas, determinamos que a altura da cerveja $h$
  e a altura da espuma $e$ satifazem aproximadamente a inequação
  $(\sqrt{e^2 + h^2 + 2 h e})^3 \leq \exp(3 \log K_0^\ast)$, onde
  $K_0^\ast$ é a altura do copo.  Esta fórmula é válida,
  aparentemente, inclusive para espécies mais pigmentadas, como {\em
  Malzbier}.  Em vista disso, e dos resultados análogos obtidos por
  A. B. Stémio em experiências com {\em Guaraná} e $x$-{\em Cola},
  conjeturamos que a fórmula pode ser aplicada (com pequenas
  modificações) também a {\em Champagne} e outros líquidos de
  composição similar.
\end{abstract}

\section{Introdução}

  As propriedades lúdicas, mnemolíticas e catalógicas de 
  soluções fraca- e medianamente concentradas de 1-metil-metanol
  em hidróxido de hidrônio aquoso ($\rm H_3O^{+} HO^{-}\cdot {\mathit n}H_2O$)
  tem sido objeto de intensos estudos experimentais por
  cientistas no mundo todo~\cite{AHU,KNU}.
  
  Durante os últimos 20 anos, os autores coordenaram uma equipe
  multidiplinar de pesquisadores, na UNICAMP e em outras instituições,
  cujo objetivo derradeiro é elucidar e quantificar a relação entre...

\begin{thebibliography}{99}

\bibitem{AHU} A. V. Aho, J. E. Hopcroft and J.  D.  Ullman, {\it The
Design and Analysis of Computer Algorithms,} Addison-Wesley (1901).

\bibitem{KNU} D. E. Knuth and L. Lamport, {\it A structural analysis
of the role of gnus and gnats in the post-modernistic, crypto-existential 
Weltanschauung of neo-liberal Tibeto-Vietnamese leaf blower operators 
as manifest in the sexual symbology of the Los Angeles Phone Directory}.
Journal of Gnu Technology, {\bf 23} (6), 12--87
(March 1996).

\end{thebibliography}

\end{document}
