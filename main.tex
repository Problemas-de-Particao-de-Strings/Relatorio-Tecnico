% Exemplo de relatório técnico do IC
% Criado por P.J.de Rezende antes do Alvorecer da História.
% Modificado em 97-06-15 e 01-02-26 por J.Stolfi.
% Last edited on 2003-06-07 21:12:18 by stolfi
% modificado em 1o. de outubro de 2008
% modificado em 2012-09-25 para ajustar o pacote UTF8. Contribuicao de
%   Rogerio Cardoso

\documentclass[11pt,twoside]{article}

%% Fontes & Template
\usepackage[utf8]{inputenc}
\usepackage[T1]{fontenc}
\usepackage[brazilian]{babel}
\usepackage{anyfontsize}
\usepackage{template/techrep-PFG-ic}

%% Outras configs
%% Hyperlinks
\usepackage{nameref, xcolor, url}
\usepackage{hyperref}
\hypersetup{
    pdftitle  = {Heurísticas para Partição Comum Mínima de Strings},
    pdfauthor = {Leonardo Rodrigues e Tiago de Paula},
    % bookmarks   = true,
    pdfpagemode = UseOutlines,
    %% Cores de Links %%
    colorlinks = true,
    linkcolor  = blue!30!black,
    urlcolor   = red!30!black,
    citecolor  = blue
}

%% Outras configs
\usepackage{amsmath}
\usepackage{amssymb}
\usepackage{amsthm}

%% Referências
\usepackage{csquotes}
\usepackage[defernumbers=true,backend=biber]{biblatex}
\addbibresource{references.bib}

%% hyperlink de tudo dentro do \cite
\DeclareCiteCommand{\cite}
    {}
    {\usebibmacro{citeindex}\printtext[bibhyperref]{[\usebibmacro{prenote}\usebibmacro{cite}\usebibmacro{postnote}]}}
    {\multicitedelim}
    {}

%% references used in footnotes
\defbibfilter{footies}{
    not type=software
}

\usepackage{enumitem, amsmath, amsthm}
\usepackage[open]{bookmark}

\makeatletter

% problemas com o ambiente 'proof' do thmbox
\LetLtxMacro\ams@proof\proof
\LetLtxMacro\ams@end@proof\endproof

\usepackage{thmtools}

\AtBeginDocument{%
    \LetLtxMacro\proof\ams@proof
    \LetLtxMacro\endproof\ams@end@proof
}

% Marcador lateral para teoremas
\newcommand*{\theorembookmark}{%
    \bookmark[
        dest = \@currentHref,
        rellevel = 1,
        keeplevel,
    ]{%
        \thmt@thmname\space\csname the\thmt@envname\endcsname
    }%
}


%% Ajusta espaços para o Display Mode %%
\newcommand{\reducemathskip}[1][0.5em]{%
    \setlength{\abovedisplayskip}{#1}%
    \setlength{\belowdisplayskip}{#1}%
    % \setlength{\abovedisplayshortskip}{#1}%
    % \setlength{\belowdisplayshortskip}{#1}%
}

%% Ambiente para Teoremas %%
\declaretheoremstyle[
    thmbox = M,
    preheadhook = \vskip1.5em,
    postheadhook = \theorembookmark\reducemathskip,
]{teorema}

\declaretheorem{theorem}[
    name = Teorema,
    % refname = {teorema,teoremas},
    % Refname= {Teorema,Teoremas},
    style = teorema,
]
\declaretheorem{lemma}[
    name  = Lema,
    style = teorema,
]
\declaretheorem{proposition}[
    name  = Proposição,
    style = teorema,
]
\declaretheorem{corollary}[
    name  = Corolário,
    style = teorema,
]

%% Ambiente para Definições %%
\declaretheorem{definition}[
    name = Definição,
    refname = {definição,definições},
    Refname= {Definição,Definições},
    %
    style = teorema,
    postheadhook = \theorembookmark\reducemathskip,
]
% Sem numeração
\declaretheorem{definition*}[
    name = Definição,
    refname = {definição,definições},
    Refname= {Definição,Definições},
    %
    style = definition,
    numbered = no,
    postheadhook = \reducemathskip
]

%% Lista de Casos %%
\newlist{casos}{enumerate}{2}
\setlist[casos]{
    wide,
    labelwidth    = {\parindent},
    listparindent = {\parindent},
    parsep        = {\parskip},
    topsep        = {0pt},
    label         = {\textbf{Caso \arabic*}:}
}
% \setlist[casos,2]{label=\textbf{Caso \arabic{casosi}\alph*}:}

%% Casos Nomeados: \item[Caso Base:] %%
\newlist{ncasos}{description}{2}
\setlist[ncasos]{
    wide,
    listparindent = {\parindent},
    parsep        = {\parskip},
    topsep        = {0pt}
}

\makeatother

\usepackage{silence}
\WarningFilter{mathptmx}{There are no bold math fonts}

\usepackage{amsmath, amssymb, bm, mathtools}
\usepackage{etoolbox, xpatch, xspace}
% \usepackage[mathcal]{euscript}
% \usepackage[scr]{rsfso}
\usepackage{mathptmx, relsize, centernot}


\makeatletter

%% Símbolo QED %%
\renewcommand{\qedsymbol}{\ensuremath{\mathsmaller\blacksquare}}

%% Marcadores de Prova: \direto, \inverso %%
\newcommand{\direto}[1][~]{\ensuremath{(\rightarrow)}#1\xspace}
\newcommand{\inverso}[1][~]{\ensuremath{(\leftarrow)}#1\xspace}

\undef\sum
%% Somatório: \sum_i^j, \bigsum_i^j %%
\DeclareSymbolFont{cmex10}{OMX}{cmex}{m}{n}
\DeclareMathSymbol{\sum@d}{\mathop}{cmex10}{"58}
\DeclareMathSymbol{\sum@t}{\mathop}{cmex10}{"50}
\DeclareMathOperator*{\sum}{\mathchoice{\sum@d}{\sum@t}{\sum@t}{\sum@t}}
\DeclareMathOperator*{\bigsum}{\mathlarger{\mathlarger{\sum@d}}}

%% Operadores de Conjunto: \pow(S), \Dom(S), \Img(S) %%
\DeclareSymbolFont{boondox}{U}{BOONDOX-cal}{m}{n}
\DeclareMathSymbol{\pow}{\mathalpha}{boondox}{"50}
\DeclareMathOperator{\Dom}{Dom}
\DeclareMathOperator{\Img}{Im}

\undef\Phi
%% Variantes gregas %%
\DeclareSymbolFont{cmr10}{OT1}{cmr}{m}{n}
\DeclareSymbolFont{cmmi10}{OML}{cmm}{m}{it}
\DeclareMathSymbol{\Phi}{\mathalpha}{cmr10}{"08}
\DeclareMathSymbol{\varpsi}{\mathalpha}{cmmi10}{"20}
\DeclareMathSymbol{\varomega}{\mathalpha}{cmmi10}{"21}

% \undef\fam
% %% Família de Conjuntos: \fam{S} %%
% \DeclareMathAlphabet{\fam}{OMS}{cmsy}{m}{n}

\undef\natural
%% Conjuntos Padrões: R, N, Z, C, Q %%
\DeclareMathOperator{\real}{\mathbb{R}}
\DeclareMathOperator{\natural}{\mathbb{N}}
\DeclareMathOperator{\integer}{\mathbb{Z}}
\DeclareMathOperator{\complex}{\mathbb{C}}
\DeclareMathOperator{\rational}{\mathbb{Q}}

%% Definição de Conjuntos: \set{ _ \mid _ } %%
\newcommand{\set}[1]{%
    \begingroup%
        \def\mid{\;\middle|\;}%
        \left\{#1\right\}
    \endgroup%
}

%% Novos Operadores: \modulo, \symdif, \grau %%
\DeclareMathOperator{\modulo}{~mod~}
\DeclareMathOperator{\symdif}{\mathrel{\triangle}}
\DeclareMathOperator{\grau}{deg}

%% Operadores Delimitados: \abs{\sum_i^j}, x \equiv y \emod{n} %%
\newcommand{\abs}[1]{{\left\lvert\,#1\,\right\rvert}}
\newcommand{\emod}[1]{\ \left(\mathrm{mod}\ #1\right)}

%% Vérices e Arestas %%
\DeclareMathOperator{\Adj}{Adj}
\DeclareMathOperator{\dist}{dist}

\makeatother

\usepackage{clrscode3e}
\usepackage{xspace}

%% KEYWORDS PADRÃO %%

\newcommand{\Para}{\kw{para}\xspace}
\newcommand{\Ate}{\kw{até}\xspace}
\newcommand{\DAte}{\kw{descendo} \Ate\xspace}
\newcommand{\Enquanto}{\kw{enquanto}\xspace}
\newcommand{\Se}{\kw{se}\xspace}
\newcommand{\Devolva}{\kw{devolva}\xspace}
\newcommand{\VaPara}{\kw{vá para}\xspace}
\newcommand{\Erro}{\kw{erro}\xspace}
\newcommand{\Nulo}{\const{Nil}\xspace}
%% Keywords não traduzidas
% \By
% \Spawn
% \Sync
% \Parfor

%% ADICIONAIS %%

\newcommand{\Cada}{\kw{cada}\xspace}
\newcommand{\Faca}{\kw{faça}\xspace}
\newcommand{\Entao}{\kw{então}\xspace}
\newcommand{\Senao}{\kw{senão}\xspace}
\newcommand{\Seja}{\kw{seja}\xspace}
\newcommand{\Sejam}{\kw{sejam}\xspace}
\newcommand{\Recebe}{\leftarrow}

\usepackage{newfloat}
\usepackage[nameinlink,noabbrev,brazilian]{cleveref}

\DeclareFloatingEnvironment[
    %listname = ⟨list name⟩,
    name = Algoritmo,
    % autorefname = pseudo,
    %legendname= ⟨name used in \legend offered by memoir⟩,
    %placement= ⟨combination of htbp⟩,
    %within= ⟨“within” counter⟩ or none,
    %chapterlistsgaps= on or off 1,
]{algorithm}
\crefname{algorithm}{algoritmo}{algoritmos}
\Crefname{algorithm}{Algoritmo}{Algoritmos}


%% Outras configs
\usepackage{amsmath}
\usepackage{amssymb}
\usepackage{amsthm}

%% TODO Notes (remover depois)
\setlength{\marginparwidth}{2cm}
\usepackage{todonotes}

% cleveref depois dos outros pacotes
\usepackage[nameinlink,noabbrev,brazilian]{cleveref}

%% Commands
\renewcommand{\S}{\mathbb{S}}
\renewcommand{\P}{\mathbb{P}}

%%%%%%%%%%%%%%%
%% Documento %%
\begin{document}

    %%% PÁGINA DE CAPA %%%%%%%%%%%%%%%%%%%%%%%%%%%%%%%%%%%%%%%%%%%%%%%
    %
    % Número do relatório
    \TRNumber{01}

    % DATA DE PUBLICAÇÃO (PARA A CAPA)
    %
    \TRYear{23}  % Dois dígitos apenas
    \TRMonth{11} % Numérico, 01-12

    % LISTA DE AUTORES PARA CAPA (sem afiliações).
    \TRAuthor{
        Leonardo de Sousa Rodrigues
        \and
        Tiago de Paula Alves
        \and
        Gabriel Siqueira
        \and
        Zanoni Dias
    }

    % TÍTULO PARA A CAPA (use \\ para forçar quebras de linha).
    \TRTitle{Heurísticas \\ MCSP}

    \TRMakeCover

    %%%%%%%%%%%%%%%%%%%%%%%%%%%%%%%%%%%%%%%%%%%%%%%%%%%%%%%%%%%%%%%%%%%%%%
    % O que segue é apenas uma sugestão - sinta-se à vontade para
    % usar seu formato predileto, desde que as margens tenham pelo
    % menos 25mm nos quatro lados, e o tamanho do fonte seja pelo menos
    % 11pt. Certifique-se também de que o título e lista de autores
    % estão reproduzidos na íntegra na página 1, a primeira depois da
    % página de capa.
    %%%%%%%%%%%%%%%%%%%%%%%%%%%%%%%%%%%%%%%%%%%%%%%%%%%%%%%%%%%%%%%%%%%%%%

    %%%%%%%%%%%%%%%%%%%%%%%%%%%%%%%%%%%%%%%%%%%%%%%%%%%%%%%%%%%%%%%%%%%%%%
    % Nomes de autores ABREVIADOS e titulo ABREVIADO,
    % para cabeçalhos em cada página.
    %
    \markboth{Leonardo, Tiago, Gabriel e Zanoni}{Heurísticas MCSP}
    \pagestyle{myheadings}

    %%%%%%%%%%%%%%%%%%%%%%%%%%%%%%%%%%%%%%%%%%%%%%%%%%%%%%%%%%%%%%%%%%%%%%
    % TÍTULO e NOMES DOS AUTORES, completos, para a página 1.
    % Use "\\" para quebrar linhas, "\and" para separar autores.
    %
    \title{Heurísticas MCSP}

    \author{
        Leonardo de Sousa Rodrigues%
        \thanks{Instituto de Computação, Universidade Estadual de Campinas, Campinas, SP.} 
        \and
        Tiago de Paula Alves\footnotemark[1]
        \and 
        Gabriel Siqueira\footnotemark[1] 
        \and 
        Zanoni Dias\footnotemark[1]
    }

    \date{}

    \maketitle

    %%%%%%%%%%%%%%%%%%%%%%%%%%%%%%%%%%%%%%%%%%%%%%%%%%%%%%%%%%%%%%%%%%%%%%

    \begin{abstract}
        Resumo bem feito sobre o PFG. Resumo bem feito sobre o PFG. Resumo bem feito sobre o PFG. Resumo bem feito sobre o PFG. Resumo bem feito sobre o PFG. Resumo bem feito sobre o PFG.
    \end{abstract}

    \section{Introdução}

        Diversos problemas de comparar strings, encontrando o menor número de operações necessárias para formar uma string a partir de outra, têm diversas aplicações em biologia computacional, em processamento de texto e em compressão de arquivos \cite{goldstein_minimum_2005}. O problema da partição comum mínima de strings (MCSP) se encaixa nessa classe de problemas, especificamente para o caso em que a única operação disponível é a reordenação de substrings.

\begin{definition}[Strings Balanceadas]
    Duas strings $A$ e $B$ são ditas \textbf{balanceadas} se o número de ocorrências dos caracteres de $A$ for igual ao dos caracteres de $B$.
\end{definition}

\begin{definition}[Partição]
    Uma sequência de strings $\part{P}$ é dita uma partição de uma string $A$ se a concatenação dos elementos de $\part{P}$ for igual a $A$. Chamamos as substrings de $A$ em $\part{P}$ de \textit{blocos}.
\end{definition}

\begin{definition}[Partição Comum]
    Para uma partição $\part{P}$ de $A$ e outra partição $\part{Q}$ de $B$, o par $\left(\part{P}, \part{Q}\right)$ é chamado de \textbf{partição comum} de $A$ e $B$ se $\part{P}$ é uma permutação de $\part{Q}$ \cite{goldstein_minimum_2005}.
\end{definition}

% \begin{enumerate}[
%     label = {\alph*)},
%     ref = \thedefinition.\alph*,
%     parsep = 0pt,
%     itemsep = 0.2em,
%     topsep = 0pt
% ]
%     \item $\abs{\S} = \abs{\P}$;
%     \label[definition]{prop1}

%     \item $S$ é o resultado da concatenação dos elementos de $\S$;

%     \item $P$ é o resultado da concatenação dos elementos de $\P$;

%     \item $\S$ é uma permutação dos elementos de $\P$.
% \end{enumerate}

O MCSP consiste em encontrar uma partição comum de tamanho mínimo (i.e. com o menor número de blocos) para duas strings balanceadas.

Ao abordar o problema pela primeira vez, alguém poderia considerar uma solução direta: testar, para todas as possíveis configurações de separação em blocos de $A$, se é possível formar $B$ a partir de uma permutação e, com isso, escolher as soluções de menor tamanho. É fácil notar, no entanto, que tal algoritmo não é polinomial, de forma que é inviável para obter uma solução do problema, mesmo com instâncias de menos de 50 caracteres. De fato, provou-se que o MCSP é um problema NP-Difícil, exceto para o caso em que cada caractere ocorre apenas uma vez em cada string \cite{goldstein_minimum_2005}. Sendo assim, heurísticas são ótimos instrumentos para encontrar soluções boas para instâncias do problema.


    \section{Heurísticas iniciais}

        Como uma primeira abordagem para a resolução do MCSP, utilizaremos duas heurísticas diretas e determinísticas, que utilizam alguma característica do problema para tentar encontrar soluções com boa qualidade: a heurística de combinação e a heurística gulosa.

\subsection{Combinação}

    A heurística de combinação consiste em um algoritmo que inicia com uma solução trivial do problema (blocos de um caractere cada) e tenta agrupar blocos mantendo uma solução válida. A implementação escolhida consiste em analisar os blocos das strings da esquerda para a direita e agrupá-los sempre que possível, como mostra o \Cref{alg:combine}. A escolha arbitrária de quais blocos serão combinados, no entanto, compromete a qualidade da solução gerada.

    \begin{algorithm}[htb]
        \caption{Heurística de combinação.} \label{alg:combine}
        \begin{codebox}
        \Procname{$\proc{Combinação}(A, B)$}
        \li $B_A \Recebe$ blocos unitários de $A$
        \li $B_B \Recebe$ blocos unitários de $B$
        \li \Para \Cada par de blocos $(b_1, b_2)$ consecutivos em $B_A$ e em $B_B$ \Faca
            \Do
        \li     $B_A \Recebe B_A$ com $b_1$ e $b_2$ combinados
        \li     $B_B \Recebe B_B$ com $b_1$ e $b_2$ combinados
            \End
        \li \Devolva $(B_A, B_B)$
        \end{codebox}
    \end{algorithm}

    \subsubsection{Análise de singletons}

        \begin{definition}[Singleton]
            Um caractere de uma string é dito singleton se a ocorrência de seu rótulo na string é 1.
        \end{definition}

        Caracteres singletons têm valor especial na resolução do MCSP: uma substring que contém tal caractere em $A$ não pode ter mais de uma substring equivalente em $B$. Com isso, é possível priorizar a combinação de blocos que contêm singletons e, com isso, melhorar a tomada de decisão do algoritmo e seus resultados. O \Cref{alg:combineS} mostra como adaptar o procedimento de combinação de blocos. Chamaremos essa modificação de combinação-S.

        \begin{algorithm}[htb]
        \caption{Heurística de combinação com análise de singletons.} \label{alg:combineS}
        \begin{codebox}
        \Procname{$\proc{Combinação-S}(A, B)$}
        \li $B_A \Recebe$ blocos unitários de $A$
        \li $B_B \Recebe$ blocos unitários de $B$
        \li \Para \Cada par de blocos $(b_1, b_2)$ consecutivos em $B_A$ e em $B_B$
        \zi tal que $b_1$ e $b_2$ possuam singletons \Faca
            \Do
        \li     $B_A \Recebe B_A$ com $b_1$ e $b_2$ combinados
        \li     $B_B \Recebe B_B$ com $b_1$ e $b_2$ combinados
            \End
        \li \Para \Cada par de blocos $(b_1, b_2)$ consecutivos em $B_A$ e em $B_B$
        \zi tal que $b_1$ ou $b_2$ possuam singletons \Faca
            \Do
        \li     $B_A \Recebe B_A$ com $b_1$ e $b_2$ combinados
        \li     $B_B \Recebe B_B$ com $b_1$ e $b_2$ combinados
            \End
        \li \Para \Cada par de blocos $(b_1, b_2)$ consecutivos em $B_A$ e em $B_B$ \Faca
            \Do
        \li     $B_A \Recebe B_A$ com $b_1$ e $b_2$ combinados
        \li     $B_B \Recebe B_B$ com $b_1$ e $b_2$ combinados
            \End
        \li \Devolva $(B_A, B_B)$
        \end{codebox}
    \end{algorithm}

\subsection{Gulosa}

    Uma outra estratégia para encontrar soluções do MCSP é a gulosa. Tal algoritmo consiste em iterativamente escolher blocos grandes que coincidam entre as duas strings e marcá-los. Em cada iteração, escolhe-se a maior substring comum entre $S$ e $P$ que não coincida com blocos já marcados em nenhuma das strings. Os dois novos blocos correspondentes são então marcados. O algoritmo progride dessa forma até que todos os caracteres das strings pertençam a blocos marcados, formando então uma partição comum. 

    \begin{algorithm}[htb]
        \caption{Heurística gulosa.} \label{alg:greedy}
        \begin{codebox}
        \Procname{$\proc{Gulosa}(A, B)$}
        \li $B_A \Recebe$ sequência com $A$ como único elemento
        \li $B_B \Recebe$ sequência com $B$ como único elemento
        \li \Enquanto existem caracteres pertencentes a blocos não marcados \Faca
            \Do
        \li     $x \Recebe$ maior substring comum a $B_A$ e $B_B$ nos blocos não marcados
        \li     quebre um bloco de $B_A$ que contém $x$ criando um novo bloco marcado com apenas $x$
        \li     quebre um bloco de $B_B$ que contém $x$ criando um novo bloco marcado com apenas $x$
            \End
        \li \Devolva $(B_A, B_B)$
        \end{codebox}
    \end{algorithm}

    \subsubsection{Maior substring comum}

        O problema de encontrar a maior substring comum entre duas strings pode ser resolvido com a ajuda de uma árvore de sufixos. Tal árvore é uma versão especializada de uma \textit{Radix tree} contendo todos os sufixos de cada string e algum marcador referente a string original.

        Com essa representação, o problema da maior substring comum se transforma em um problema de encontrar o ancestral comum mais recente entre duas folhas de origens distintas na árvore. Esse ancestral comum representa o maior prefixo comum entre todos os sufixos das duas strings, sendo portanto a maior subtring comum entre elas.

        A árvore de sufixos pode ser adaptada também para manter substrings de múltiplas string distintas, permitindo a implementação eficiente da maior substring comum em uma coleção de strings. Esse adaptação não foi implementada nesse trabalho.


    \section{Representação por grafo}

        Outra forma de construir soluções boas para o MCSP é utilizando meta-heurísticas, i.e. algoritmos de otimização que utilizam aleatorização em conjunto com busca local \cite[p.~4]{yang_nature-inspired_2010}. Antes de aplicar algum dos vários algoritmos conhecidos, no entanto, é necessário representar as instâncias do problema utilizando uma estrutura de dados eficiente e adequada para tais procedimentos. Encontrar uma boa representação para problemas complexos, como esse, é uma etapa desafiadora, mas que pode simplificar muito a resolução.

Desenvolvemos uma representação baseada em um grafo de substrings comuns \cite{ferdous_solving_2017}, adaptando a representação para arestas múltiplas e ignorando as substrings de tamanho unitário.

\begin{definition}[Grafo de substrings comuns]
    Dado um par de strings balanceadas $A$ e $B$ de tamanho $n = \abs{A} = \abs{B}$, temos o grafo de substrings comuns $G_{A,B}(V,E,\varphi)$ de $A$ em $B$ onde:

    \begin{enumerate}[
        label = {\alph*)},
        ref = \thedefinition.\alph*,
        parsep = 0pt,
        itemsep = 0.2em,
        topsep = 0pt
    ]
        \item os vértices são dados pelos índices da string $A$: \[
            V = \set{1, \ldots, n} = I_n
        \]

        \item as arestas representam um bloco não-trivial (mais de um caractere) e suas posições como substring de $A$ e $B$: \[
            E = \set{e_{p, q, k} \mid A_p \ldots A_{p + k - 1} = B_q \ldots B_{q + k - 1} ~\text{ e }~ k \geq 2}
        \]

        \item as arestas conectam os caracteres inicial e final de $A$ no bloco: \[
            \varphi\left(e_{p, q, k}\right) = \set{p, p + k - 1}
        \]
    \end{enumerate}
\end{definition}

Note que cada par de bloco representado por uma aresta pode ser utilizado na construção de uma partição comum entre as strings. Arestas paralelas indicam que existe mais de uma substring em $B$ igual à substring de $A$ formada pelos caracteres entre suas extremidades. Um exemplo da representação pode ser visto na \Cref{fig:grafo}.

É importante ressaltar que blocos de apenas 1 caractere não são considerados. Como as strings de entrada são balanceadas e os blocos das arestas são idênticos, partindo de uma partição comum incompleta, podemos completá-la com todos os caracteres não contidos na partição comum como blocos unitários. Não considerar tais blocos reduz consideravelmente o número de arestas da representação, e consequentemente possibilita algoritmos mais eficientes.

\begin{figure}
    \centering
    \begin{subfigure}{\textwidth}
        \centering
        \import{images}{grafo-abba.tikz}

        \caption{\textcolor{red}{Grafo representando a instância (\texttt{abba}, \texttt{abab}).}}
    \end{subfigure}
    \begin{subfigure}{\textwidth}
        \centering
        \import{images}{grafo-abab.tikz}

        \caption{\textcolor{red}{Grafo representando a instância (\texttt{abab}, \texttt{abba}).}}
    \end{subfigure}

    \caption{\textcolor{red}{Grafo representando as instâncias (\texttt{abba}, \texttt{abab}) e (\texttt{abab}, \texttt{abba}).}}
    \label{fig:grafo}
\end{figure}

\subsection{Construindo partições} \label{sec:construindo-particoes}

    A partir de uma sequência ordenada das arestas da representação de uma instância do MCSP, podemos construir partições da seguinte forma: para cada aresta, na ordem dada, incluímos os dois blocos à partição comum se nenhum deles coincide com caracteres contidos em blocos que já foram incluídos. Ao fim do processo, cria-se blocos unitários para todos os caracteres não contidos em um bloco. Dessa forma, uma permutação do conjunto de arestas do grafo representa uma única partição comum entre as strings de entrada do problema.

    É válido ressaltar duas propriedades dessa representação. Primeiramente, nem todas as possíveis partições possuem uma permutação correspondente, já que os blocos unitários são omitidos. Tais partições, no entanto, não são as mais interessantes para o MCSP, já que buscamos partições menores.

    Em segundo lugar, note que a partição comum formada por uma permutação das arestas não é única: podemos trocar de posição arestas vizinhas que não são utilizadas, de forma que o mesmo resultado é obtido.

\subsection{Aplicação da representação}

    Com a construção dessa representação, agora o MCSP se reduz a encontrar uma maneira de ordenar as arestas de forma que a partição comum resultante seja a menor possível. Isso significa que o reduzimos a um problema de permutação. Algoritmos de otimização que atuam em um espaço de busca contínuo podem ser utilizados para esse tipo de problema aplicando um sistema de pesos que assumem valores reais \cite[p.~661]{marti_handbook_2018}. Relacionamos o vetor de arestas com um vetor de pesos de mesmo tamanho, de forma que a ordenação das arestas pelos valores dos pesos correspondentes forma a permutação correspondente àqueles valores. Como cada peso é uma componente no espaço de busca, podemos nos referir ao vetor de pesos também como posição (e à sua variação, como velocidade).


    \section{Particle swarm optimization}

        Utilizando a representação por grafo desenvolvida, optamos por implementar o \textit{Particle Swarm Optimization} (PSO). Esse algoritmo foi descoberto durante simulações de modelos sociais de animais \cite{kennedy_particle_1995} e baseia-se no comportamento de enxames e bandos de aves -- mas também de cardumes ou até grupos humanos \cite[p.~7]{yang_nature-inspired_2010} \todo{página 7? \cite{yang_nature-inspired_2010}}. Cada agente busca localmente em seus arredores por posições de qualidade, participando também da comunicação do grupo para que todos saibam qual é a melhor encontrada até o momento. O PSO ganhou popularidade nas últimas duas décadas devido ao agradável balanço que proporciona entre a eficiência da busca por soluções e a facilidade de implementação e de adaptação ao problema em que é aplicado \cite[p.~640]{marti_handbook_2018}.

De forma mais prática, o algoritmo consiste em um conjunto de partículas em que cada uma possui uma posição atual e mantém gravada a melhor posição que encontrou até o momento. O enxame, que contém as partículas, também mantém a melhor posição global encontrada. A cada iteração, a posição de uma partícula é atualizada de acordo com as componentes escolhidas, como explica a \Cref{sec:movimento}.

Para aplicar ao contexto do MCSP, iniciamos o algoritmo construindo a representação por grafo da instância e obtendo o vetor de arestas. Posições iniciais para a quantidade de partículas desejadas são geradas, conforme explica a \Cref{sec:posicoes}. Para medir a qualidade da solução referente a uma partícula, utilizamos a posição como vetor de pesos para obter uma ordenação das arestas do grafo. Essa ordenação é analisada para avaliar quantos blocos existe naquela solução, sem necessariamente gerar as partições correspondentes, já que isso apenas aumentaria o tempo de execução. Finalmente calculamos a função objetivo, a ser maximizada, dada pela diferença entre o número de caracteres da string e o número de blocos da partição. Ao fim do número de iterações desejado, utilizamos a melhor posição global para construir a solução do problema.

\subsection{Posições iniciais} \label{sec:posicoes}

    Foram implementadas diferentes formas de gerar posições iniciais para as partículas visando diversificar o enxame, cobrindo ao máximo o espaço de busca, mas também incluindo soluções boas como ponto de partida para algumas partículas. As componentes das posições, apesar de não serem limitadas durante a execução, são inicializadas no intervalo $[-1,1]$.

    Uma das formas de incluir boas soluções é utilizar uma partição comum resultante da aplicação de outra heurística para gerar um vetor de pesos correspondente. Isso é feito verificando quais arestas do grafo são compatíveis com a partição comum dada, i.e. quais arestas têm seu bloco naquela partição. Feito isso, geramos um peso para cada aresta compatível no intervalo $[-1,0)$ e para cada não compatível no intervalo $[0, 1]$, de forma que, ao serem ordenadas por esses pesos, a mesma partição inicial pode ser construída, já que as arestas compatíveis serão consideradas antes das outras. Os pesos são gerados aleatoriamente no intervalo dado, permitindo a geração de diversas posições inciais a partir da mesma partição comum.
    \todo[inline]{(Gabriel) Vocês só garantem que o processo vai gerar a mesma partição se a partição inicial é maximal (o que é verdade para os duas heuristicas, então ok nessa parte) e se vocês verificam se tem pares de blocos que podem ser combinados seguindo a ordem das arestas, ou seja, se tem uma aresta que equivale a combinação de blocos da partição, vocês combinam esses blocos.}

    Outra forma de propiciar posições inciais de qualidade é incluindo ordenações do vetor de arestas por tamanho do bloco, obtendo partições comuns similares à heurística gulosa. Para isso, são gerados pesos aleatórios em $[-1,1]$ e arranjados de forma a priorizar as arestas com maiores blocos.

    Por fim, a geração totalmente aleatória dos pesos uniformemente em $[-1,1]$, apesar de ingênua, gerou bons resultados por si só e também mostrou-se útil quando combinada às outras formas de construção de posições iniciais. De forma geral, consolidamos a criação das posições iniciais do enxame utilizando os procedimentos descritos:

    \begin{enumerate}[
        label = {\alph*)},
        parsep = 0pt,
        itemsep = 0.2em,
        topsep = 0pt
    ]
        \item A partir do resultado da heurística gulosa
        \item A partir do resultado da heurística de combinação-S
        \item A partir da ordenação das arestas por tamanho do bloco
        \item Geração totalmente aleatória
    \end{enumerate}

    Para cada partícula, uma das formas é escolhida com probabilidade proporcional a um peso designado a ela. Utilizaremos a ordem das formas de geração para simplificar a notação que representa a distribuição das partículas. Notaremos $D = (1, 2, 0, 2)$, por exemplo, para indicar que uma distribuição de probabilidades foi usada com $20\%$ de chance do resultado da heurística gulosa ser utilizado, $40\%$ de chance do resultado da heurística combinação-S ser utilizado, $0\%$ de chance da ordenação por tamanho do bloco ser utilizada e $40\%$ da geração totalmente aleatória.

\subsection{Movimento das partículas} \label{sec:movimento}

    No início de cada iteração do algoritmo, as partículas devem se movimentar pelo espaço de busca para buscar melhores soluções. Esse movimento pode ser composto por diferentes componentes. Nas origens do PSO, utilizava-se duas componentes: uma em direção à melhor solução encontrada pela própria partícula (melhor individual) e outra em direção à melhor solução encontrada pela vizinhança da partícula. A vizinhança pode ser definida de diversas maneiras, como por distância ou como apenas uma vizinhança global \cite{bratton_defining_2007}.

    Experimentos posteriores com o PSO consideraram diferentes componentes para atualização das posições, como a implementação de um sistema de inércia e um limite máximo para a velocidade da partícula \cite{shi_parameter_1998}.

    Nossa implementação do algoritmo aqui apresentada para o MCSP utiliza uma combinação das componentes originais, com uma vizinhança global, e um comportamento estocástico \cite[p.~642]{marti_handbook_2018}. Seja $\vec{p}_j$ a posição de uma partícula na iteração $j$, $\vec{p}_L$ a sua melhor posição até agora e $\vec{p}_G$ a melhor posição encontrada pelo enxame. Define-se estaticamente $k_L$, $k_G$ e $k_E$, como parâmetros do algoritmo. Se $R_L \sim \text{Uniforme}(0, k_L)$, $R_G \sim \text{Uniforme}(0, k_G)$ e se $\vec{R}_E$ representa um vetor de componentes geradas uniformemente de forma aleatória em $[-k_E, k_E]$, então

    \[
        \vec{p}_{j+1} = \vec{p}_j + \left(\vec{p}_L - \vec{p}_j\right) \cdot R_L + \left(\vec{p}_G - \vec{p}_j\right) \cdot R_G + \vec{R}_E
    \]


    \section{Resultados}

    \section{Discussão}

    \newpage
    \printbibliography

\end{document}
