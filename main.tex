% Exemplo de relatório técnico do IC
% Criado por P.J.de Rezende antes do Alvorecer da História.
% Modificado em 97-06-15 e 01-02-26 por J.Stolfi.
% Last edited on 2003-06-07 21:12:18 by stolfi
% modificado em 1o. de outubro de 2008
% modificado em 2012-09-25 para ajustar o pacote UTF8. Contribuicao de
%   Rogerio Cardoso

\documentclass[11pt,twoside]{article}

%% Fontes & Template
\usepackage[utf8]{inputenc}
\usepackage[T1]{fontenc}
\usepackage[brazilian]{babel}
\usepackage{anyfontsize}
\usepackage{template/techrep-PFG-ic}

%% Pacotes simples
\usepackage[section]{placeins}
\usepackage{caption}
\usepackage{subcaption}
\usepackage{enumitem}
\usepackage{booktabs}
\usepackage{import}
\usepackage{pgf}
\usepackage{tikz}
\usepackage{siunitx}
\usepackage{multirow}

%% Pacotes complicados
%% Hyperlinks
\usepackage{nameref, xcolor, url}
\usepackage{hyperref}
\hypersetup{
    pdftitle  = {Heurísticas para Partição Comum Mínima de Strings},
    pdfauthor = {Leonardo Rodrigues e Tiago de Paula},
    % bookmarks   = true,
    pdfpagemode = UseOutlines,
    %% Cores de Links %%
    colorlinks = true,
    linkcolor  = blue!30!black,
    urlcolor   = red!30!black,
    citecolor  = blue
}

%% Outras configs
\usepackage{amsmath}
\usepackage{amssymb}
\usepackage{amsthm}

%% Referências
\usepackage{csquotes}
\usepackage[defernumbers=true,backend=biber]{biblatex}
\addbibresource{references.bib}

%% hyperlink de tudo dentro do \cite
\DeclareCiteCommand{\cite}
    {}
    {\usebibmacro{citeindex}\printtext[bibhyperref]{[\usebibmacro{prenote}\usebibmacro{cite}\usebibmacro{postnote}]}}
    {\multicitedelim}
    {}

%% references used in footnotes
\defbibfilter{footies}{
    not type=software
}

\usepackage{amsmath, amssymb, bm, mathtools}
\usepackage{etoolbox, xpatch, xspace}
% \usepackage[mathcal]{euscript}
% \usepackage[scr]{rsfso}
\usepackage{mathptmx, relsize, centernot}


\makeatletter

%% Símbolo QED %%
\renewcommand{\qedsymbol}{\ensuremath{\mathsmaller\blacksquare}}

%% Marcadores de Prova: \direto, \inverso %%
\newcommand{\direto}[1][~]{\ensuremath{(\rightarrow)}#1\xspace}
\newcommand{\inverso}[1][~]{\ensuremath{(\leftarrow)}#1\xspace}

\undef\sum
%% Somatório: \sum_i^j, \bigsum_i^j %%
\DeclareSymbolFont{cmex10}{OMX}{cmex}{m}{n}
\DeclareMathSymbol{\sum@d}{\mathop}{cmex10}{"58}
\DeclareMathSymbol{\sum@t}{\mathop}{cmex10}{"50}
\DeclareMathOperator*{\sum}{\mathchoice{\sum@d}{\sum@t}{\sum@t}{\sum@t}}
\DeclareMathOperator*{\bigsum}{\mathlarger{\mathlarger{\sum@d}}}

%% Operadores de Conjunto: \pow(S), \Dom(S), \Img(S) %%
\DeclareSymbolFont{boondox}{U}{BOONDOX-cal}{m}{n}
\DeclareMathSymbol{\pow}{\mathalpha}{boondox}{"50}
\DeclareMathOperator{\Dom}{Dom}
\DeclareMathOperator{\Img}{Im}

\undef\Phi
%% Variantes gregas %%
\DeclareSymbolFont{cmr10}{OT1}{cmr}{m}{n}
\DeclareSymbolFont{cmmi10}{OML}{cmm}{m}{it}
\DeclareMathSymbol{\Phi}{\mathalpha}{cmr10}{"08}
\DeclareMathSymbol{\varpsi}{\mathalpha}{cmmi10}{"20}
\DeclareMathSymbol{\varomega}{\mathalpha}{cmmi10}{"21}

% \undef\fam
% %% Família de Conjuntos: \fam{S} %%
% \DeclareMathAlphabet{\fam}{OMS}{cmsy}{m}{n}

\undef\natural
%% Conjuntos Padrões: R, N, Z, C, Q %%
\DeclareMathOperator{\real}{\mathbb{R}}
\DeclareMathOperator{\natural}{\mathbb{N}}
\DeclareMathOperator{\integer}{\mathbb{Z}}
\DeclareMathOperator{\complex}{\mathbb{C}}
\DeclareMathOperator{\rational}{\mathbb{Q}}

%% Definição de Conjuntos: \set{ _ \mid _ } %%
\newcommand{\set}[1]{%
    \begingroup%
        \def\mid{\;\middle|\;}%
        \left\{#1\right\}
    \endgroup%
}

%% Novos Operadores: \modulo, \symdif, \grau %%
\DeclareMathOperator{\modulo}{~mod~}
\DeclareMathOperator{\symdif}{\mathrel{\triangle}}
\DeclareMathOperator{\grau}{deg}

%% Operadores Delimitados: \abs{\sum_i^j}, x \equiv y \emod{n} %%
\newcommand{\abs}[1]{{\left\lvert\,#1\,\right\rvert}}
\newcommand{\emod}[1]{\ \left(\mathrm{mod}\ #1\right)}

%% Vérices e Arestas %%
\DeclareMathOperator{\Adj}{Adj}
\DeclareMathOperator{\dist}{dist}

\makeatother

\usepackage{enumitem, amsmath, amsthm}
\usepackage[open]{bookmark}

\makeatletter

% problemas com o ambiente 'proof' do thmbox
\LetLtxMacro\ams@proof\proof
\LetLtxMacro\ams@end@proof\endproof

\usepackage{thmtools}

\AtBeginDocument{%
    \LetLtxMacro\proof\ams@proof
    \LetLtxMacro\endproof\ams@end@proof
}

% Marcador lateral para teoremas
\newcommand*{\theorembookmark}{%
    \bookmark[
        dest = \@currentHref,
        rellevel = 1,
        keeplevel,
    ]{%
        \thmt@thmname\space\csname the\thmt@envname\endcsname
    }%
}


%% Ajusta espaços para o Display Mode %%
\newcommand{\reducemathskip}[1][0.5em]{%
    \setlength{\abovedisplayskip}{#1}%
    \setlength{\belowdisplayskip}{#1}%
    % \setlength{\abovedisplayshortskip}{#1}%
    % \setlength{\belowdisplayshortskip}{#1}%
}

%% Ambiente para Teoremas %%
\declaretheoremstyle[
    thmbox = M,
    preheadhook = \vskip1.5em,
    postheadhook = \theorembookmark\reducemathskip,
]{teorema}

\declaretheorem{theorem}[
    name = Teorema,
    % refname = {teorema,teoremas},
    % Refname= {Teorema,Teoremas},
    style = teorema,
]
\declaretheorem{lemma}[
    name  = Lema,
    style = teorema,
]
\declaretheorem{proposition}[
    name  = Proposição,
    style = teorema,
]
\declaretheorem{corollary}[
    name  = Corolário,
    style = teorema,
]

%% Ambiente para Definições %%
\declaretheorem{definition}[
    name = Definição,
    refname = {definição,definições},
    Refname= {Definição,Definições},
    %
    style = teorema,
    postheadhook = \theorembookmark\reducemathskip,
]
% Sem numeração
\declaretheorem{definition*}[
    name = Definição,
    refname = {definição,definições},
    Refname= {Definição,Definições},
    %
    style = definition,
    numbered = no,
    postheadhook = \reducemathskip
]

%% Lista de Casos %%
\newlist{casos}{enumerate}{2}
\setlist[casos]{
    wide,
    labelwidth    = {\parindent},
    listparindent = {\parindent},
    parsep        = {\parskip},
    topsep        = {0pt},
    label         = {\textbf{Caso \arabic*}:}
}
% \setlist[casos,2]{label=\textbf{Caso \arabic{casosi}\alph*}:}

%% Casos Nomeados: \item[Caso Base:] %%
\newlist{ncasos}{description}{2}
\setlist[ncasos]{
    wide,
    listparindent = {\parindent},
    parsep        = {\parskip},
    topsep        = {0pt}
}

\makeatother

\usepackage{silence}
\WarningFilter{mathptmx}{There are no bold math fonts}

\usepackage{clrscode3e}
\usepackage{xspace}

%% KEYWORDS PADRÃO %%

\newcommand{\Para}{\kw{para}\xspace}
\newcommand{\Ate}{\kw{até}\xspace}
\newcommand{\DAte}{\kw{descendo} \Ate\xspace}
\newcommand{\Enquanto}{\kw{enquanto}\xspace}
\newcommand{\Se}{\kw{se}\xspace}
\newcommand{\Devolva}{\kw{devolva}\xspace}
\newcommand{\VaPara}{\kw{vá para}\xspace}
\newcommand{\Erro}{\kw{erro}\xspace}
\newcommand{\Nulo}{\const{Nil}\xspace}
%% Keywords não traduzidas
% \By
% \Spawn
% \Sync
% \Parfor

%% ADICIONAIS %%

\newcommand{\Cada}{\kw{cada}\xspace}
\newcommand{\Faca}{\kw{faça}\xspace}
\newcommand{\Entao}{\kw{então}\xspace}
\newcommand{\Senao}{\kw{senão}\xspace}
\newcommand{\Seja}{\kw{seja}\xspace}
\newcommand{\Sejam}{\kw{sejam}\xspace}
\newcommand{\Recebe}{\leftarrow}

\usepackage{newfloat}
\usepackage[nameinlink,noabbrev,brazilian]{cleveref}

\DeclareFloatingEnvironment[
    %listname = ⟨list name⟩,
    name = Algoritmo,
    % autorefname = pseudo,
    %legendname= ⟨name used in \legend offered by memoir⟩,
    %placement= ⟨combination of htbp⟩,
    %within= ⟨“within” counter⟩ or none,
    %chapterlistsgaps= on or off 1,
]{algorithm}
\crefname{algorithm}{algoritmo}{algoritmos}
\Crefname{algorithm}{Algoritmo}{Algoritmos}


%% TODO Notes (remover depois)
\setlength{\marginparwidth}{2cm}
\usepackage{todonotes}
\usepackage{soul}

%%%%%%%%%%%%%%%
%% Documento %%
\begin{document}

    %%% PÁGINA DE CAPA %%%%%%%%%%%%%%%%%%%%%%%%%%%%%%%%%%%%%%%%%%%%%%%
    %
    % Número do relatório
    \TRNumber{61}

    % DATA DE PUBLICAÇÃO (PARA A CAPA)
    %
    \TRYear{23}  % Dois dígitos apenas
    \TRMonth{11} % Numérico, 01-12

    % LISTA DE AUTORES PARA CAPA (sem afiliações).
    \TRAuthor{Leonardo de Sousa Rodrigues
        \and Tiago de Paula Alves
        \and Gabriel Siqueira
        \and Zanoni Dias
    }

    % TÍTULO PARA A CAPA (use \\ para forçar quebras de linha).
    \TRTitle{Heurísticas para Partição Comum Mínima de Strings}

    \TRMakeCover

    %%%%%%%%%%%%%%%%%%%%%%%%%%%%%%%%%%%%%%%%%%%%%%%%%%%%%%%%%%%%%%%%%%%%%%
    % O que segue é apenas uma sugestão - sinta-se à vontade para
    % usar seu formato predileto, desde que as margens tenham pelo
    % menos 25mm nos quatro lados, e o tamanho do fonte seja pelo menos
    % 11pt. Certifique-se também de que o título e lista de autores
    % estão reproduzidos na íntegra na página 1, a primeira depois da
    % página de capa.
    %%%%%%%%%%%%%%%%%%%%%%%%%%%%%%%%%%%%%%%%%%%%%%%%%%%%%%%%%%%%%%%%%%%%%%

    %%%%%%%%%%%%%%%%%%%%%%%%%%%%%%%%%%%%%%%%%%%%%%%%%%%%%%%%%%%%%%%%%%%%%%
    % Nomes de autores ABREVIADOS e titulo ABREVIADO,
    % para cabeçalhos em cada página.
    %
    \markboth{Leonardo, Tiago, Gabriel e Zanoni}{Heurísticas MCSP}
    \pagestyle{myheadings}

    %%%%%%%%%%%%%%%%%%%%%%%%%%%%%%%%%%%%%%%%%%%%%%%%%%%%%%%%%%%%%%%%%%%%%%
    % TÍTULO e NOMES DOS AUTORES, completos, para a página 1.
    % Use "\\" para quebrar linhas, "\and" para separar autores.
    %
    \title{Heurísticas para Partição Comum Mínima de Strings}

    \author{
        Leonardo de Sousa Rodrigues%
            \thanks{Instituto de Computação, Universidade Estadual de Campinas, Campinas, SP.}
        \and Tiago de Paula Alves\footnotemark[1]
        \and Gabriel Siqueira\footnotemark[1]
        \and Zanoni Dias\footnotemark[1]
    }

    \date{}

    \maketitle

    %%%%%%%%%%%%%%%%%%%%%%%%%%%%%%%%%%%%%%%%%%%%%%%%%%%%%%%%%%%%%%%%%%%%%%

    \begin{abstract}
        O problema de Partição Comum Mínima de Strings (MCSP) é utilizado na comparação de strings com aplicações em biologia computacional. Boas heurísticas têm grande valor para esse problema, já que ele foi provado ser NP-Difícil. Além de apresentar implementações para heurísticas conhecidas da literatura, desenvolvemos uma representação por grafo eficiente para instâncias do MCSP, reduzindo-o a um problema de permutação e permitindo a aplicação de algoritmos de otimização para buscar soluções. O \textit{Particle Swarm Optimization} (PSO) foi adaptado para esta representação e foi capaz de não só melhorar significativamente o resultado das outras heurísticas utilizadas, mas também encontrar boas soluções de forma independente, mostrando-se uma meta-heurística promissora, principalmente para instâncias com poucas repetições de caracteres. Esse trabalho sugere a utilização da representação por grafo com outros métodos de otimização para o MCSP.
    \end{abstract}

    \raggedbottom

    \section{Introdução}
        \import{text}{intro}

    \section{Heurísticas Iniciais}
        \import{text}{iniciais}

    \section{Representação por Grafo}
        \import{text}{grafo}

    \section{Particle Swarm Optimization}
        \import{text}{pso}

    \section{Resultados e Discussão}
        \import{text}{resultados}

    \section{Conclusão}
        \import{text}{conclusao}

    \printbibliography[filter=footies]

\end{document}
