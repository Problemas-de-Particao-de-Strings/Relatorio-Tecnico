\usepackage{amsmath, amssymb, bm, mathtools}
\usepackage{etoolbox, xpatch, xspace}
% \usepackage[mathcal]{euscript}
% \usepackage[scr]{rsfso}
\usepackage{mathptmx, relsize, centernot}
% \usepackage{newtxtext, newtxmath}


\makeatletter

%% Símbolo QED %%
\renewcommand{\qedsymbol}{\ensuremath{\mathsmaller\blacksquare}}

%% Marcadores de Prova: \direto, \inverso %%
\newcommand{\direto}[1][~]{\ensuremath{(\rightarrow)}#1\xspace}
\newcommand{\inverso}[1][~]{\ensuremath{(\leftarrow)}#1\xspace}

\undef\sum
%% Somatório: \sum_i^j, \bigsum_i^j %%
\DeclareSymbolFont{cmex10}{OMX}{cmex}{m}{n}
\DeclareMathSymbol{\sum@d}{\mathop}{cmex10}{"58}
\DeclareMathSymbol{\sum@t}{\mathop}{cmex10}{"50}
\DeclareMathOperator*{\sum}{\mathchoice{\sum@d}{\sum@t}{\sum@t}{\sum@t}}
\DeclareMathOperator*{\bigsum}{\mathlarger{\mathlarger{\sum@d}}}

%% Operadores de Conjunto: \pow(S), \Dom(S), \Img(S) %%
\DeclareSymbolFont{boondox}{U}{BOONDOX-cal}{m}{n}
\DeclareMathSymbol{\pow}{\mathalpha}{boondox}{"50}
\DeclareMathOperator{\Dom}{Dom}
\DeclareMathOperator{\Img}{Im}

\undef\Phi
%% Variantes gregas %%
\DeclareSymbolFont{cmr10}{OT1}{cmr}{m}{n}
\DeclareSymbolFont{cmmi10}{OML}{cmm}{m}{it}
\DeclareMathSymbol{\Phi}{\mathalpha}{cmr10}{"08}
\DeclareMathSymbol{\varpsi}{\mathalpha}{cmmi10}{"20}
\DeclareMathSymbol{\varomega}{\mathalpha}{cmmi10}{"21}

\undef\part
%% Família de Conjuntos ou Partição: \part{S} %%
\DeclareMathAlphabet{\part}{OMS}{cmsy}{m}{n}

\undef\natural
%% Conjuntos Padrões: R, N, Z, C, Q %%
\DeclareMathOperator{\real}{\mathbb{R}}
\DeclareMathOperator{\natural}{\mathbb{N}}
\DeclareMathOperator{\integer}{\mathbb{Z}}
\DeclareMathOperator{\complex}{\mathbb{C}}
\DeclareMathOperator{\rational}{\mathbb{Q}}

%% Definição de Conjuntos: \set{ _ \mid _ } %%
\newcommand{\set}[1]{%
    \begingroup%
        \def\mid{\;\middle|\;}%
        \left\{#1\right\}
    \endgroup%
}

%% Novos Operadores: \modulo, \symdif, \grau %%
\DeclareMathOperator{\modulo}{~mod~}
\DeclareMathOperator{\symdif}{\mathrel{\triangle}}
\DeclareMathOperator{\grau}{deg}

%% Operadores Delimitados: \abs{\sum_i^j}, x \equiv y \emod{n} %%
\newcommand{\abs}[1]{{\left\lvert\,#1\,\right\rvert}}
\newcommand{\emod}[1]{\ \left(\mathrm{mod}\ #1\right)}

%% Vérices e Arestas %%
\DeclareMathOperator{\Adj}{Adj}
\DeclareMathOperator{\dist}{dist}

\makeatother
