Utilizando a representação por grafo desenvolvida, optamos por implementar o \textit{Particle Swarm Optimization} (PSO). Esse algoritmo foi descoberto durante simulações de modelos sociais de animais~\cite{kennedy_particle_1995} e baseia-se no comportamento de enxames e bandos de aves -- mas também de cardumes ou até grupos humanos~\cite[p.~7]{yang_nature-inspired_2010}. Cada agente busca localmente em seus arredores por posições de qualidade, participando também da comunicação do grupo para que todos saibam qual é a melhor encontrada até o momento. O PSO ganhou popularidade nas últimas duas décadas devido ao balanço que proporciona entre a eficiência da busca por soluções e a facilidade de implementação e de adaptação ao problema em que é aplicado~\cite[p.~640]{marti_handbook_2018}.

De forma mais prática, o algoritmo consiste em um conjunto de partículas em que cada uma possui uma posição atual e mantém gravada a melhor posição que encontrou até o momento. O enxame, que contém as partículas, também mantém a melhor posição global encontrada. A cada iteração, a posição de uma partícula é atualizada de acordo com as componentes escolhidas, como explicado na \Cref{sec:movimento}.

Para aplicar ao contexto do MCSP, iniciamos o algoritmo construindo a representação por grafo da instância e obtendo o vetor de arestas. Posições iniciais para a quantidade de partículas desejadas são geradas, conforme explicado na \Cref{sec:posicoes}. Para medir a qualidade da solução referente a uma partícula, utilizamos a posição como vetor de pesos para obter uma ordenação das arestas do grafo. Essa ordenação é analisada para avaliar quantos blocos existe naquela solução, sem necessariamente gerar as partições correspondentes, já que isso apenas aumentaria o tempo de execução. Finalmente calculamos a função objetivo, a ser maximizada, dada pela diferença entre o número de caracteres da string e o número de blocos da partição. Ao fim do número de iterações desejado, utilizamos a melhor posição global para construir a solução do problema.

\subsection{Posições Iniciais} \label{sec:posicoes}

    Foram implementadas diferentes formas de gerar posições iniciais para as partículas visando diversificar o enxame, cobrindo ao máximo o espaço de busca, mas também incluindo soluções boas como ponto de partida para algumas partículas. As componentes das posições, apesar de não serem limitadas durante a execução, são inicializadas no intervalo $[-1,1]$.

    Uma das formas de incluir boas soluções é utilizar uma partição comum resultante da aplicação de outra heurística para gerar um vetor de pesos correspondente. Isso é feito verificando quais arestas do grafo são compatíveis com a partição comum dada, i.e., quais arestas têm seu bloco naquela partição. Feito isso, geramos um peso para cada aresta compatível no intervalo $[-1,0)$ e para cada não compatível no intervalo $[0, 1]$, de forma que, ao serem ordenadas por esses pesos, a partição construída terá todos os blocos não-unitários da partição inicial, já que as arestas compatíveis serão consideradas antes das outras. Os pesos são gerados aleatoriamente no intervalo dado, permitindo a geração de diversas posições iniciais a partir da mesma partição comum.

    Outra forma de propiciar posições iniciais de qualidade é incluindo ordenações do vetor de arestas por tamanho do bloco, obtendo partições comuns similares à heurística gulosa. Para isso, são gerados pesos aleatórios no intervalo $[-1,1]$ e arranjados de forma a priorizar as arestas com maiores blocos.

    Por fim, a geração totalmente aleatória dos pesos uniformemente em $[-1,1]$, apesar de ingênua, gerou bons resultados por si só e também mostrou-se útil quando combinada às outras formas de construção de posições iniciais. De forma geral, consolidamos a criação das posições iniciais do enxame utilizando os procedimentos descritos:

    \begin{enumerate}[
        label = {\alph*)},
        parsep = 0pt,
        itemsep = 0.2em,
        topsep = 0pt
    ]
        \item A partir do resultado da heurística gulosa
        \item A partir do resultado da heurística de combinação-S
        \item A partir da ordenação das arestas por tamanho do bloco
        \item Geração totalmente aleatória
    \end{enumerate}

    Para cada partícula, uma das formas é escolhida com probabilidade proporcional a um peso designado a ela. Utilizaremos a ordem das formas de geração para simplificar a notação que representa a distribuição das partículas. Notaremos $D = (1, 2, 0, 2)$, por exemplo, para indicar que uma distribuição de probabilidades foi usada com $20\%$ de chance do resultado da heurística gulosa ser utilizado, $40\%$ de chance do resultado da heurística combinação-S ser utilizado, $0\%$ de chance da ordenação por tamanho do bloco ser utilizada e $40\%$ da geração totalmente aleatória.

\subsection{Movimento das Partículas} \label{sec:movimento}

    No início de cada iteração do algoritmo, as partículas devem se movimentar pelo espaço de busca para buscar melhores soluções. Esse movimento pode ser composto por diferentes componentes. Nas origens do PSO, utilizava-se duas componentes: uma em direção à melhor solução encontrada pela própria partícula (melhor individual) e outra em direção à melhor solução encontrada pela vizinhança da partícula. A vizinhança pode ser definida de diversas maneiras, como por distância ou como apenas uma vizinhança global~\cite{bratton_defining_2007}.

    Experimentos posteriores com o PSO consideraram diferentes componentes para atualização das posições, como a implementação de um sistema de inércia e um limite máximo para a velocidade da partícula~\cite{shi_parameter_1998}.

    Nossa implementação do algoritmo aqui apresentada para o MCSP utiliza uma combinação das componentes originais, com uma vizinhança global, e um comportamento estocástico~\cite[p.~642]{marti_handbook_2018}. Seja $\vec{p}_j$ a posição de uma partícula na iteração $j$, $\vec{p}_L$ a sua melhor posição até agora e $\vec{p}_G$ a melhor posição encontrada pelo enxame. Define-se estaticamente $k_L$, $k_G$ e $k_E$, como parâmetros do algoritmo. Se $R_L \sim \text{Uniforme}(0, k_L)$, $R_G \sim \text{Uniforme}(0, k_G)$ e se $\vec{R}_E$ representa um vetor de componentes geradas uniformemente de forma aleatória em $[-k_E, k_E]$, então

    \[
        \vec{p}_{j+1} = \vec{p}_j + \left(\vec{p}_L - \vec{p}_j\right) \cdot R_L + \left(\vec{p}_G - \vec{p}_j\right) \cdot R_G + \vec{R}_E
    \]

\subsection{Exemplo de Execução}

    \begin{figure}
        \centering
        \import{images}{grafo-cdbcea.tikz}
    
        \caption{Grafo representando a instância (\texttt{cdbcea}, \texttt{bceacd}). Vetor equivalente $\left[e_{1,5,2}, e_{3,1,2}, e_{3,1,3}, e_{3,1,4}, e_{4,2,2}, e_{4,2,3}, e_{5,3,2}\right]$}
        \label{fig:grafo-pso}
    \end{figure}
