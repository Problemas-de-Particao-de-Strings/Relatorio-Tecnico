
\subsection{Combinação}

    A heurística de combinação consiste em um algoritmo que inicia com uma solução trivial do problema (blocos de um caractere cada) e tenta agrupar blocos mantendo uma solução válida. A implementação escolhida consiste em analisar os blocos das strings da esquerda para a direita e agrupá-los sempre que possível, como mostra o \cref{alg:combine}. A escolha genérica de quais blocos serão combinados, no entanto, compromete a qualidade da solução gerada.

    \begin{algorithm}
        \caption{Heurística de combinação.} \label{alg:combine}
        \begin{codebox}
        \Procname{$\proc{Combinação}(S, P)$}
        \li $B_S \Recebe$ blocos unitários de $S$
        \li $B_P \Recebe$ blocos unitários de $P$
        \li \Para \Cada par de blocos $(b_1, b_2)$ consecutivos em $B_S$ e em $B_P$ \Faca
            \Do
        \li     $B_S \Recebe B_S$ com $b_1$ e $b_2$ combinados
        \li     $B_P \Recebe B_P$ com $b_1$ e $b_2$ combinados
            \End
        \li \Devolva $(B_S, B_P)$
        \end{codebox}
    \end{algorithm}

    \subsubsection{Análise de singletons}
    
        \begin{definition}[Singleton]
            Um caractere de uma string é dito singleton se é a única ocorrência daquele rótulo na string.
        \end{definition}

        Caracteres singletons têm valor especial na resolução do MCSP: uma substring que contém tal caractere em $S$ não pode ter mais de uma substring equivalente em $P$. Com isso, é possível priorizar a combinação de blocos que contêm singletons e, com isso, melhorar a tomada de decisão do algoritmo e seus resultados. O \cref{alg:combineS} mostra como adaptar o procedimento de combinação de blocos.

        \begin{algorithm}
        \caption{Heurística de combinação com análise de singletons.} \label{alg:combineS}
        \begin{codebox}
        \Procname{$\proc{CombinaçãoComSingletons}(S, P)$}
        \li $B_S \Recebe$ blocos unitários de $S$
        \li $B_P \Recebe$ blocos unitários de $P$
        \li \Para \Cada par de blocos $(b_1, b_2)$ consecutivos em $B_S$ e em $B_P$
        \zi tal que $b_1$ e $b_2$ possuam singletons \Faca
            \Do
        \li     $B_S \Recebe B_S$ com $b_1$ e $b_2$ combinados
        \li     $B_P \Recebe B_P$ com $b_1$ e $b_2$ combinados
            \End
        \li \Para \Cada par de blocos $(b_1, b_2)$ consecutivos em $B_S$ e em $B_P$
        \zi tal que $b_1$ ou $b_2$ possuam singletons \Faca
            \Do
        \li     $B_S \Recebe B_S$ com $b_1$ e $b_2$ combinados
        \li     $B_P \Recebe B_P$ com $b_1$ e $b_2$ combinados
            \End
        \li \Para \Cada par de blocos $(b_1, b_2)$ consecutivos em $B_S$ e em $B_P$ \Faca
            \Do
        \li     $B_S \Recebe B_S$ com $b_1$ e $b_2$ combinados
        \li     $B_P \Recebe B_P$ com $b_1$ e $b_2$ combinados
            \End
        \li \Devolva $(B_S, B_P)$
        \end{codebox}
    \end{algorithm}

\subsection{Guloso}

    Uma outra estratégia para encontrar soluções do MCSP é a gulosa. Tal algoritmo consiste em iterativamente escolher blocos grandes que coincidam entre as duas strings e marcá-los. Em cada iteração, escolhe-se a maior substring comum entre $S$ e $P$ que não coincida com blocos já marcados em nenhuma das strings. Os dois novos blocos correspondentes são então marcados. O algoritmo progride dessa forma até que todos os caracteres das strings pertençam a blocos marcados, formando então uma partição.

    \begin{algorithm}
        \caption{Heurística gulosa.} \label{alg:greedy}
        \begin{codebox}
        \Procname{$\proc{Gulosa}(S, P)$}
        \li $B_S \Recebe$ sequência com $S$ como único elemento
        \li $B_P \Recebe$ sequência com $P$ como único elemento
        \li \Enquanto existem caracteres pertencentes a blocos não marcados \Faca
            \Do
        \li     $x \Recebe$ maior substring comum a $B_S$ e $B_P$ nos blocos não marcados
        \li     quebre um bloco de $B_S$ que contém $x$ e marque o bloco de $x$
        \li     quebre um bloco de $B_P$ que contém $x$ e marque o bloco de $x$
            \End
        \li \Para \Cada par de blocos $(b_1, b_2)$ consecutivos em $B_S$ e em $B_P$ \Faca
            \Do
        \li     $B_S \Recebe B_S$ com $b_1$ e $b_2$ combinados
        \li     $B_P \Recebe B_P$ com $b_1$ e $b_2$ combinados
            \End
        \li \Devolva $(B_S, B_P)$
        \end{codebox}
    \end{algorithm}

    \subsubsection{Maior substring comum}

        \todo[inline]{Descrever radix tree e algoritmo LCS}
