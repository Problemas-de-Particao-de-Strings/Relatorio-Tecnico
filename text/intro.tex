Problemas de comparação de strings, que encontram o menor número de operações necessárias para formar uma string a partir de outra, têm diversas aplicações em biologia computacional, em processamento de texto e em compressão de arquivos~\cite{goldstein_minimum_2005}. O problema da partição comum mínima de strings (MCSP) se encaixa nessa classe de problemas, especificamente para o caso em que a única operação disponível é a reordenação de substrings.

Seja uma string $A$, com $\abs{A}$ caracteres. Denotamos por $A_i$ o i-ésimo caractere de $A$. O conjunto de caracteres distintos de $A$ é chamado de \textit{alfabeto} de $A$. Utilizamos o termo \textit{rótulo} para se referir aos elementos do alfabeto e o termo caractere para se referir aos elementos da string~\cite[p.~17]{siqueira_heuristicas_2021}.

\begin{definition}[Ocorrência]
    A ocorrência de um rótulo $\alpha$ em uma string $A$ é o número de cópias de $\alpha$ presentes em $A$.
\end{definition}

\begin{definition}[Strings Balanceadas]
    Duas strings são ditas balanceadas se possuem o mesmo alfabeto e a ocorrência de todos os caracteres é igual nas duas strings.
\end{definition}

\begin{definition}[Partição]
    Uma sequência de strings $\part{P}$ é dita uma partição de uma string $A$ se a concatenação dos elementos de $\part{P}$ for igual a $A$. Chamamos as substrings de $A$ em $\part{P}$ de \textit{blocos}. O tamanho de $\part{P}$ é dado pelo seu número de blocos e denotado por $\abs{\part{P}}$.
\end{definition}

\begin{definition}[Partição Comum]
    Para uma partição $\part{P}$ de $A$ e outra partição $\part{Q}$ de $B$, o par $\left(\part{P}, \part{Q}\right)$ é chamado de \textbf{partição comum} de $A$ e $B$ se $\part{P}$ é uma permutação de $\part{Q}$~\cite{goldstein_minimum_2005}. O tamanho de uma partição comum $\left(\part{P}, \part{Q}\right)$ é dado por $\abs{\left(\part{P}, \part{Q}\right)} = \abs{\part{P}} = \abs{\part{Q}}$.
\end{definition}

O \textbf{MCSP} (do inglês, \textit{minimum common string partition}) consiste em encontrar uma partição comum de tamanho mínimo para duas strings balanceadas. Ao abordar o problema pela primeira vez, alguém poderia considerar uma solução direta: testar, para todas as possíveis configurações de separação em blocos de $A$, se é possível formar $B$ a partir de uma permutação e, com isso, escolher as soluções de menor tamanho. É fácil notar, no entanto, que tal algoritmo não é polinomial, sendo inviável obter uma solução do problema dessa forma, mesmo com instâncias de menos de 50 caracteres. De fato, provou-se que o MCSP é um problema NP-Difícil, exceto para o caso em que cada caractere ocorre apenas uma vez em cada string~\cite{goldstein_minimum_2005}. Sendo assim, heurísticas são ótimos instrumentos para encontrar soluções boas para instâncias do problema.

Nas últimas décadas, diversos trabalhos fizeram progresso para encontrar melhores soluções e garantir aproximações para o MCSP. Em 2004, \textcite{chrobak_greedy_2004} mostraram que a heurística gulosa tem aproximação em $\Omega(n^{0.43})$ e em $O(n^{0.69})$. Em 2005, \textcite{goldstein_minimum_2005} provaram que o problema é NP-difícil, propondo uma 1.1037-aproximação para o 2-MCSP e uma 4-aproximação para o 3-MCSP. O $k$-MCSP é uma versão restrita do MCSP em que a ocorrência máxima da entrada é no máximo $k$. Ainda em 2005, \textcite{chen_assignment_2005} elaboraram o algoritmo SOAR com uma 1.5-aproximação para o 2-SMCSP (versão com sinais do problema). Em 2007, \textcite{cormode_string_2007} apresentaram uma aproximação em $O(\log n \log^* n)$ e \textcite{kolman_reversal_2007} mostraram uma $4k$-aproximação para o $k$-MCSP. Além disso, \textcite{mandoiu_novel_2007} introduziu a ideia de explorar os rótulos de ocorrência 1 para melhorar a heurística gulosa e \textcite{kolman_approximating_2007} modificaram essa heurística para garantir aproximação em $O(k^2)$. Em 2010, \textcite{jiang_minimum_2012} propuseram um algoritmo parametrizado com tempo em $O^*((k!)^{2c})$, em que $c$ é o tamanho da solução.

Em 2013, \textcite{bulteau_minimum_2014} conseguiram um algoritmo parametrizado com tempo em $c^{21k^2} \text{poli}(n)$ e, no mesmo ano, \textcite{bulteau_fixed-parameter_2013} atingiram um algoritmo parametrizado com tempo em $O(k^{2c'} \cdot cn)$, em que $c'$ é o número de blocos que contém pelo menos um caractere replicado. Em 2014, \textcite{goldstein_quick_2014} sugeriram uma versão da heurística gulosa em tempo linear. No ano seguinte, \textcite{blum_mathematical_2015} formularam o problema utilizando programação linear inteira (PLI) e outra formulação com PLI foi feita por \textcite{blum_computational_2016}. Em 2016, \textcite{blum_construct_2016} construíram uma meta-heurística que utiliza soluções exatas de instâncias menores do problema. Esse algoritmo foi adaptado em 2018 para instâncias maiores~\cite{blum_minimum_2018}.

Em 2016, \textcite{ferdous_solving_2017} implementaram uma meta-heurística que se baseia no método da colônia de formigas para encontrar soluções para o MCSP, utilizando uma representação por grafo de substrings comuns das strings de entrada do problema. Recentemente, em 2021, \textcite{siqueira_signed_2023} apresentaram uma $2k$-aproximação para o problema.

Nas seções desse relatório, primeiramente mostraremos e explicaremos heurísticas diretas e determinísticas para o MCSP: a de combinação e a gulosa. Depois, discutiremos uma representação por grafo para as entradas do problema, que permite a aplicação de algoritmos de otimização. Em seguida, mostraremos detalhadamente uma implementação do \textit{Particle Swarm Optimization} utilizado em conjunto com a representação desenvolvida. Por fim, discutiremos os resultados obtidos e o tempo de execução das heurísticas.
